\documentclass[11pt]{report}
\usepackage{outline}
\usepackage{pmgraph}
\usepackage[normalem]{ulem}
\usepackage[utf8]{inputenc}
\usepackage[french]{babel}
\title{\textbf{GIF-4100 : Vision numérique\\ Projet de trimestre \\La feuille électronique}}
\author{Olivier Beaulieu,\\Frédéric Bernard,\\Kento Otomo-Lauzon}
%% \date{\oldstylenums{00}/\oldstylenums{00}/\oldstylenums{00}}
%--------------------Make usable space all of page
\setlength{\oddsidemargin}{0in}
\setlength{\evensidemargin}{0in}
\setlength{\topmargin}{0in}
\setlength{\headsep}{-.25in}
\setlength{\textwidth}{6.5in}
\setlength{\textheight}{8.5in}
%--------------------Indention
\setlength{\parindent}{1cm}

\begin{document}
%--------------------Title Page
\maketitle
 

\newpage

\section{Solution proposée}
La solution proposée est constituée d'un projecteur et d'une caméra. Le
projecteur est disposé de manière à illuminer une table, et la caméra sert à
détecter la position de feuilles blanches et d'un doigt.





\section{Performances}

\section{Améliorations possibles}
% Utiliser une seconde caméra pour identifier quand on touche vs quand on touche pas
% Faire un « calibrage » sur le seuillage de couleur pour trouver le bras




\end{document}
