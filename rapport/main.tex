\documentclass[11pt]{report}
\usepackage{outline}
\usepackage{pmgraph}
\usepackage[normalem]{ulem}
\usepackage[utf8]{inputenc}
\usepackage[french]{babel}
\title{\textbf{GIF-4100 : Vision numérique\\ Projet de trimestre \\La feuille électronique}}
\author{Olivier Beaulieu,\\Frédéric Bernard,\\Kento Otomo-Lauzon}
%% \date{\oldstylenums{00}/\oldstylenums{00}/\oldstylenums{00}}
%--------------------Make usable space all of page
\setlength{\oddsidemargin}{0in}
\setlength{\evensidemargin}{0in}
\setlength{\topmargin}{0in}
\setlength{\headsep}{-.25in}
\setlength{\textwidth}{6.5in}
\setlength{\textheight}{8.5in}
%--------------------Indention
\setlength{\parindent}{1cm}

\begin{document}
%--------------------Title Page
\maketitle
 

\newpage

\section{Solution proposée}
La solution proposée est constituée d'un projecteur et d'une caméra. Le
projecteur est disposé de manière à illuminer une table, et la caméra sert à
détecter la position de feuilles blanches et d'un doigt.

\subsection{Fonctionnement général}
Le système permet à un utilisateur, en utilisant son doigt, de dessiner sur une
feuille en temps réel. En localisant la position du doigt et de la feuille
blanche sur la table, le système superpose le tracé effectué par l'utilisateur
sur la feuille. Dans le cas où plusieurs feuilles sont présentes sur la table,
le dessin est dupliqué sur chaque feuille. Le dessin étant défini de manière
relative à la feuille, celui-ci demeure à l'endroit, et ce, même si
l'utilisateur déplace la feuille ou lui fait effectuer une rotation.

Finalement, le dessin est visible depuis une application web, laquelle est
accessible depuis un ordinateur.

Dans un contexte de télétravail, on pourrait imaginer un scénario d'utilisation
selon lequel deux collègues sont disposés autour de la table en ayant chacun une
feuille. Les membres physiquement présents à la réunion pourraient participer à
une séance de remue-méninges en dessinant sur leur propre feuille, et le
résultat serait automatiquement reflété sur les feuilles des autres
participants. De plus, des participants présents à distance pourraient effectuer
le même exercice en se connectant à l'application web. 

\subsection{}

\section{Performances}

\section{Améliorations possibles}
% Utiliser une seconde caméra pour identifier quand on touche vs quand on touche pas
% Faire un « calibrage » sur le seuillage de couleur pour trouver le bras




\end{document}
